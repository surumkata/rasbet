\chapter{Restrições do Projeto}  
\newcounter{resnum}
\setcounter{resnum}{1}

\section{Restrições Obrigatórias da Solução}
\begin{itemize}
    \item \textbf{Descrição:} O apostador tem acesso a uma listagem de desportos, e após a seleção dessa modalidade, a uma listagem de jogos. 
    
    \textbf{Justificação:} De forma a poder apostar na equipa favorita, caso seja um desporto coletivo, ou em um único participante, no caso de desporto individual, o apostador tem que ter acesso a toda a lista com os jogadores intervenientes.
    
    \item \textbf{Descrição:} A aplicação tem de ser um \textit{web site}. 
    
    \textbf{Justificação:}  A aplicação deverá poder ser usada em qualquer lugar, a partir do seu computador ou \textit{smartphone}.
\end{itemize}

\section{Ambiente de Implementação do Sistema}
A Empresa irá alocar os recursos necessários para manter a aplicação.

As sessões de teste da aplicação serão realizadas nos \textit{browsers} Google Chrome, Microsoft Edge e Safari.

A Empresa quer monitorizar o desenvolvimento da aplicação via github.


\section{Restrições Prazo/Agendamento}
\begin{itemize}
    \item \textbf{Descrição:} O documento presente terá se ser entregue numa fase inicial, até dia 04 de novembro de 2022.
    
    \textbf{Justificação:} De forma a poder ser avaliado o estado do projeto numa fase inicial, é necessário que seja feita uma entrega que contenha a primeira fase deste projeto, que abarca a contextualização e a definição dos requisitos da solução.
\end{itemize}

\section{Restrições Orçamento}
\begin{itemize}
    \item \textbf{Descrição:} O orçamento total para o desenvolvimento do projeto é de 20 000€ (vinte mil euros), durante um período de 4 meses.
    
    \textbf{Justificação:} A equipa responsável pelo desenvolvimento do projeto é constituída por cnco engenheiros de \textit{software}. Para além de ter em conta os salários dos elementos, é preciso também a compra de um domínio, bem como a de servidores para alojar todos os dados da aplicação.
\end{itemize}
